% !TEX root = ./main.tex

\section{まとめ} \label{section:conclusion}

% TODO: conclusion
本論文ではJuliaプログラムの型安全性を向上させる手法として,
JuliaのJITに使用されている型推論ルーチンを利用してプログラムを型レベルで抽象解釈することで,
データ型エラーを静的に検出する型プロファイラTypeProfiler.jlを提案した.

既存の型推論ルーチンを利用することで,比較的簡単に型プロファイラを実装することができ,
また抽象解釈の収束性とスケーラビリティを保つために行われているヒューリスティックを
そのまま用いることができた.
一方でそのヒューリスティックはパフォーマンスの向上を目的とした型推論ルーチンの開発において
培われたものであり,型プロファイリングにおいては,
そのdesign spaceを型エラーの検出という観点からまた改めて見直す必要がある.

今回提案した型プロファイラは,
プロファイリングの正確性と実行速度のトレードオフを見極めつつ,今後さらに実装を改善することで,
Juliaプログラムの型安全性を向上させる上での有効なツールとなることが期待される.
